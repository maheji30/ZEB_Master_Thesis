\chapter{Figuras}
\label{chap:figuras}


 
 
 \section{Tipos de figuras y sus editores}
\subsection{Configuración de ambiente y posicionamiento}
 En la Figura~\ref{fig:IER} se muestra una captura de la página web del IER-UNAM.
 \begin{figure}
 \centering
 \includegraphics[scale=0.2]{ier_homepage}
 \caption{	
 Captura tomada de la página web del IER-UNAM.
 \label{fig:IER}
 }
 \end{figure}
 
 La figura se incluyó con el siguiente código
 \begin{verbatim}
 \begin{figure}
 \centering
 \includegraphics[scale=0.2]{ier_homepage}
 \caption{	
 Captura tomada de la página web del IER-UNAM.
 \label{fig:IER}}
 \end{figure}
 \end{verbatim}
 
Una práctica muy común que veo es que usan 
 \begin{verbatim}
 \begin{figure}[!ht]
 \end{verbatim}
 que le está diciendo a latex forzar a poner la figura en la misma página que el texto y en la parte superior de la página. Te recomiendo no usar esto y darle oportunidad a latex de acomodar las figuras de acuerdo a su criterio, además conforme tengas más texto, las figuras irán cambiando su lugar y se verá mejor.
\subsection{Figuras en pdf}
\subsection{Figuras en png/jpg}
\subsection{Figuras en epslatex}
\subsection{Figuras con gnuplot}
\subsection{Figuras en tex}
\subsection{Recomendaciones generales}