\chapter{Introducción}
\label{chap:introduccion}

\section{General objective}

To implement a methodology to evaluate if a building is zero energy for buildings in the design process or in buildings already built at CU-UNAM.


\section{Specific objectives}

\begin{itemize}

\item To provide a literature review about the concept of NZEB (Net-Zero Energy Building).

\item To develop a model of a building already built  at CU-UNAM to assess the meeting of the NZEB definitions.

\item To Implement the PMV and ePMV within the methodology.

\item To propose and evaluate strategies for passive design, energy efficiency and renewable generation that allow meeting the NZEB definitions.

\item To obtain a guide that can be taken as a reference in future building construction projects at CU-UNAM so that they can achieve NZEB status.

\end{itemize}


\section{Energy Buildings: A Critical Look at the Definition}

A net zero-energy building (NZEB) is a construction that uses efficiency gains in order to reduce the building's energy consumption so the rest of the energy needs can be supplied by renewable technologies. When the generation is greater than the building's loads, excess electricity is exported to the grid and it can offset later energy use. Due the use of storage technologies still being limited, the use of the grid is substantial to achieve ZEB.

There are four commonly used definitions for ZEB: net zero site energy (site-NZEB), net zero source energy (source-NZEB), net zero energy costs (cost-NZEB), and net zero energy emissions (emission-NZEB). A Net Zero Site Energy produces at least as much energy as it uses in a year, when accounted for at the site. A Net Zero Source Energy produces at least as much energy as it uses in a year, when accounted for at the source. Source energy refers to the primary energy used to generate and deliver the energy to the site. To calculate a building's total source energy, imported and exported energy is multiplied by the appropriate site-to-source conversion factors. In the Net Zero Energy Costs, the amount of money the utility pays the building owner for the energy the building exports to the grid is at least equal to the amount the owner pays the utility for the energy services and energy used over the year. Net Zero Energy Emissions produces at least as much emissions-free renewable energy as it uses from emissions-producing energy sources. 

The use of each one depends on the project goals that the building owner has. For example, building owners typically care about energy costs, so they can use the Net Zero Energy Costs definition. The governments of the countries are concerned with national energy numbers, and are typically interested in primary or source energy. A building designer may be interested in site energy use for energy code requirements. Finally, those who are concerned about pollution may be interested in reducing emissions. The zero energy definition affects how buildings are designed to achieve the goal.
\newpage
\section{Extension of the PMV model to non-air-conditioned buildings in warm climates}
The PMV model agrees well with high-quality field studies in buildings with HVAC systems, situated in cold, temperate and warm climates, studied during both summer and winter. In non-air-conditioned buildings in warm climates, occupants may sense the warmth as being less severe than the PMV predicts. The main factor that explains this phenomenon is the expectations of the occupants, which are people who have been living in hot environments. They are likely to judge a warm environment as less unacceptable than people who are used to air conditioning. This can be expressed by an expectation factor, \textit{e}. This factor varies between 0.5 and 1, the latter being the appropriate value for buildings with air conditioning. For buildings without air conditioning, it is assumed that the expectation factor depends on the duration of warm weather during the year and whether there are other buildings in the region with air conditioning. Thus, if the weather is hot all year or most of the year and there are few or no other buildings with air conditioning, \textit{e} may be 0.5, whereas it may be 0.7 if there are many other buildings with air-conditioning. A second factor contributing to this difference is the estimated activity. In many office field studies, metabolic rate is estimated with a questionnaire that identifies the percentage of time the person was sitting, standing, or walking. This approach fails to acknowledge the fact that people, when they feel hot, unconsciously tend to slow down their activity. They adapt to the hot environment by slowing down their metabolic rate. The lower rate in hot environments should be recognized by inserting a reduced metabolic rate when calculating PMV. Subsequent studies were conducted in various offices, where the recorded metabolic rates were reduced by 6.7\% for each PMV scale unit above neutrality. In conclusion, to calculate de ePMV the metabolic rate is reduced by 6.7\% for each PMV scale unit. With this reduced metabolic rate, PMV is recalculated. Finally, the PMV obtained is multiplied by the appropriate \textit{e}. The extended PMV model agrees well with quality field studies in non-air-conditioned buildings of three continents.

\newpage

\section{Lessons Learned from Case Studies of Six High-Performance Building}
NREL studied six buildings to understand the issues related to the design, construction, operation, and evaluation of the current generation of low energy commercial buildings. Each case study developed a list of lessons learned and recommendations relevant to that unique building. These buildings and the lessons learned from them help inform a set of best practices, beneficial design elements, technologies, and techniques that should be encouraged in future buildings, as well as pitfalls to be avoided. 

The 30\% of total energy in the commercial buildings is for lighting, so there are two practices that were applied in the six buildings in order to achieve energy savings in this end use. The first is by daylighting design, which means that is necessary to design strategically the building so it can allow the access of daylight and reduce the consume of electric light. In the six buildings, this was done through the use of clerestories, elongated east-west axes, south glazing, skylights and sidelit and toplit daylight. The second practice is daylight and lighting controls, which means that it is necessary to install the appropriate equipment in order to turn off electric lighting in response to natural light or when there is no occupancy. The six buildings used daylight sensors and motion sensors. For Oberlin, Zion, TTF, and BigHorn, the total lighting saving represented the largest source of the total energy saving. Total lighting energy savings ranged from 93\% in the BigHorn warehouse to 30\% at CBF. 

Natural ventilation can also provide significant energy savings. Four of the buildings used natural ventilation for cooling and ventilation. The systems at Zion and BigHorn were mostly successful. Zion was designed to be entirely naturally ventilated with a stack effect that is primarily driven by cool air provided by the cooltowers. The EMS mechanically operates windows in the clerestory. Natural downdraft cooltowers that require no fans are also considered natural ventilation. Lower manually operated windows complete the scheme. BigHorn has motor-actuated clerestory windows that are controlled either by the EMS or by wall switches. The front doors, which open during normal business hours in the summer, provide supply air to complete the scheme. 

The use of PV systems to meet the ZEB goals is substantial. During night-time hours when the Oberlin or Cambria PV system was in standby mode, the inverters consumed electricity. This losses accounted for 7\% to 37\% of the total PV output. An automatic circuit that disconnects the PV system from the grid when the PV system is down and reconnects when the PV system is operational should be implemented. Also PV panels should be located where they will not be shaded. Each PV array experienced shading, resulting in a range of energy penalties from 2\% to 44\% output reduction.

\section{Criteria of sustainable construction - UNAM}
In September 2020, the General Direction of Works and Conservation of UNAM published this document as a guide that establishes technical, preventive, corrective and safety measures in the construction and remodelling of university buildings, with the aim of minimizing the negative effects that impact the environment, taking advantage of natural resources and elements in a sustainable manner.  The objective of this document is to provide the UNAM dependencies with guidelines for proper management in the use, exploitation and saving of water and electricity from the project and during the construction and operation of the buildings. The document contains important information about the buildings at CU-UNAM, such as lighting levels, dimensions of sanitary and education spaces, characteristics of the envelope and the standards involved.

\section{Net-Zero Energy Buildings: the influence of definition on greenhouse gas emissions}

This study compares the effectiveness of the four NZEB definitions to reduce operational emissions of a building. A simple geometry located in two different cities, that is, Toronto and Miami, with four different energy behaviours, which are simulated in OpenStudio, is used. It was found that, for both locations, the use of a cost-NZEB balance leads to lower emissions, a reduction of 102-145$\%$ for Toronto and 99-117$\%$ for Miami. The emission-ZEB definition, contrary to its designation, is the least effective in reducing GHG emissions, at 86$\%$ for Toronto and 89-94$\%$ for Miami.

Important elements to consider: The article shows the most important equations to develop each of the definitions.


\section{A review of net zero energy buildings in hot and humid climates: Experience learned from 34 case study buildings}

This document summarizes the design features and technology choices obtained through analysis of 34 NZEB cases in hot and humid climates around the world. The design features and technologies are divide into five groups: architectural design and envelope; heating, ventilation and air-conditioning (HVAC); lighting; plug load equipment; and renewable energy technologies. 

The case studies indicated NZEBs are always equipped with an advanced HVAC system and energy-efficient ventilation strategies. Twenty-four cases use natural ventilation strategies to introduce free cooling to NZEBs and reduce HVAC system energy use. Solar photovoltaics are the most common renewable energy technologies adopted by NZEBs cases, with 30 cases found in this research.

The study found that passive design and technologies such as daylighting and natural ventilation are often adopted for NZEBs in hot and humid climates, together with other energy efficient and renewable energy technologies. 

The Xingye building case demonstrated that natural ventilation could reduce the building's cooling energy from 2.5 kWh/m2 to less than 0.5 kWh/m2 per month — an 80\% cooling energy reduction in the shoulder seasons compared with the summer season. With daylighting, the Xingye building demonstrated that monthly lighting energy intensity can be controlled less than 0.5 kWh/m2.

\section{Ten questions about zero energy buildings: A state of the art review.}
This document has the aim of identifying, developing and understanding the main characteristics of zero energy buildings. For this, a review of the state of the art of the topic is carried out, where 97 articles considered to be of greater relevance were selected, in the period from 2006 to 2020. The methodology consisted of an analysis of these texts based on ten questions formulated to address the topic. The questions refer to definitions (P1), sustainability (P2), technologies involved (P3), emissions (P5), energy (P4) (P6) (P7), regulations (P8), climate change (P9) and future projections (P10). The work allows us to conclude that the ZEBs are integrated in a holistic way in the transformation towards a renewable and sustainable future in terms of energy solutions and have the potential to be implemented in different geographical and climatic positions.


Important elements to consider: Consumption for heating and cooling is usually the highest in buildings, so the reviewed literature proposes achieving savings in ZEB, between $\%$25 and $\%$50, respectively, limiting both at 30 $\frac{kWh}{m^{2}*year}$.


\section{A Guide to Zero Energy and Zero Energy Ready K–12 Schools}

This document describes the steps to create a ZE school. These steps serve as a guide to ensure that a school achieves its ZE design goal and maintains its ZE status after it is occupied and operational. The topics covered in the steps are: Assessment of the needs of a building, integration of zero energy goals in the design, management of goals in the design and operation phases, performance evaluation, etc. Each step also includes an experience about the step. These brief anecdotes were provided by the participating school districts and offer a brief synopsis of the district's experience with that step.


\section{Analysis of the environmental, energy and economic effects between a sustainable construction model and a conventional one in Querétaro}

The document focuses on performing an analysis of the environmental, energy and economic effects resulting from the construction of buildings using a sustainability model compared to a conventional model, in order to obtain a cost indicator when using a sustainable construction model to a house and see if it is possible that it is not more than 15$\%$ against using a traditional construction model. 3 traditional housing models and 3 sustainable housing models were proposed to make comparisons between them. The energy and thermal simulation of all the cases was carried out in EnergyPlus. Comparing the best-equipped sustainable housing model with the worst-equipped traditional housing model, the percentage increase in cost was 14.75\$%$.

Important elements to consider: Site-to-Source conversion factors, CO2 emission factors, Occupancy schedules and residential loads and take into account inflation and NPV in the calculation of savings.

\section{Methodology for the validation of thermal simulations of a real building}

This document shows a methodology to validate thermal simulations of a real building. Comparison metrics and tolerance ranges, both obtained from the literature, are also included. An IER building is used as a case study. Two cases are proposed, a base case where sun protection is used and another one where it is not used. The results include graphs for qualitative comparison of the measurement and the two simulated cases. A quantitative comparison was made using the most common metrics described in the literature review. In both comparisons, the result obtained for case 2 is better than for case 1, this suggests that the solar protections of the building are absorbing heat, due to the red color, and are transmitting heat by conduction and radiation to the interior of the building.


Important elements to consider: It is likely that for the ZEB thesis data will need to be collected from the building of interest, for which a correct validation will need to be carried out. This thesis can be a good guide in case I don't remember very well what the methodology was like.


\section{Thermal comfort studies}

The aim of this document is to explore the evaluation of thermal comfort that the use of passive strategies and low energy consumption provides to occupants in naturally ventilated buildings. In this work, a methodology is proposed for the validation of thermal simulations, whose results show that the building model obtained with the calibration process is applicable to different times of the year, occupancy and ventilation conditions, as well as obtaining accurate results of the thermal comfort assessment using night ventilation strategies and change in the absorptivity of the envelope. The thesis also proposes a methodology for the evaluation of thermal, acoustic and light comfort in the design stage of the new IER building, for which it is proposed to use the extended predicted mean vote method.


Important elements to consider: The validation methodology of the bachelor thesis is used. The document contains important information about the PMVe.



