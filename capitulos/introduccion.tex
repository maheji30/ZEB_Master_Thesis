\chapter{Introducción}
\label{chap:introduccion}

 


\section{Energy Buildings: A Critical Look at the Definition}

The aim of this document is to explore the concept of zero energy: what it means and why a clear and measurable definition is needed. Four well-documented definitions are studied. The recommended activities to achieve a ZEB are also shown in a hierarchical manner. The definitions are applied to a set of low energy buildings and the influence each definition has on the design of the ZEB is explained. Subsequently, the advantages and disadvantages to achieve each definition are shown. This study shows that the way the zero energy goal is defined affects the choices designers apply to the building to achieve this goal.


Important elements to consider: In order to calculate the savings the designed ZEB, it is necessary a base case for comparison, which could be found in ASHRAE 2001, in the section on Conventional Buildings Minimally Compliant with the Energy Code.
\section{Net-Zero Energy Buildings: A Classification System Based on Renewable Energy Supply Options}

\section{Lessons Learned from Case Studies of Six High-Performance Building}

\section{Definition of a Zero Net Energy Community}

\section{A Common Definition for Zero Energy Buildings}

\section{Net-Zero Energy Buildings: the influence of definition on greenhouse gas emissions}

This study compares the effectiveness of the four NZEB definitions to reduce operational emissions of a building. A simple geometry located in two different cities, that is, Toronto and Miami, with four different energy behaviors, which are simulated in OpenStudio, is used. It was found that, for both locations, the use of a NZEC balance leads to lower emissions, a reduction of 102-145$\%$ for Toronto and 99-117$\%$ for Miami. The NZEE definition, contrary to its designation, is the least effective in reducing GHG emissions, at 86$\%$ for Toronto and 89-94$\%$ for Miami.

Important elements to consider: The article shows the most important equations to develop each of the definitions.


\section{A review of net zero energy buildings in hot and humid climates: Experience learned from 34 case study buildings}

\section{Ten questions about zero energy buildings: A state of the art review.}

This document has the aim of identifying, developing and understanding the main characteristics of zero energy buildings. For this, a review of the state of the art of the topic is carried out, where 97 articles considered to be of greater relevance were selected, in the period from 2006 to 2020. The methodology consisted of an analysis of these texts based on ten questions formulated to address the topic. The questions refer to definitions (P1), sustainability (P2), technologies involved (P3), emissions (P5), energy (P4) (P6) (P7), regulations (P8), climate change (P9) and future projections (P10). The work allows us to conclude that the ZEBs are integrated in a holistic way in the transformation towards a renewable and sustainable future in terms of energy solutions and have the potential to be implemented in different geographical and climatic positions.


Important elements to consider: Consumption for heating and cooling is usually the highest in buildings, so the reviewed literature proposes achieving savings in ZEB, between $\%$25 and $\%$50, respectively, limiting both at 30 $\frac{kWh}{m^{2}*year}$.


\section{A Guide to Zero Energy and Zero Energy Ready K–12 Schools}

This document describes the steps to create a ZE school. These steps serve as a guide to ensure that a school achieves its ZE design goal and maintains its ZE status after it is occupied and operational. The topics covered in the steps are: Assessment of the needs of a building, integration of zero energy goals in the design, management of goals in the design and operation phases, performance evaluation, etc. Each step also includes an experience about the step. These brief anecdotes were provided by the participating school districts and offer a brief synopsis of the district's experience with that step.


\section{Analysis of the environmental, energy and economic effects between a sustainable construction model and a conventional one in Querétaro}

The document focuses on performing an analysis of the environmental, energy and economic effects resulting from the construction of buildings using a sustainability model compared to a conventional model, in order to obtain a cost indicator when using a sustainable construction model to a house and see if it is possible that it is not more than 15$\%$ against using a traditional construction model. 3 traditional housing models and 3 sustainable housing models were proposed to make comparisons between them. The energy and thermal simulation of all the cases was carried out in EnergyPlus. Comparing the best-equipped sustainable housing model with the worst-equipped traditional housing model, the percentage increase in cost was 14.75\$%$.

Important elements to consider: Site-to-Source conversion factors, CO2 emission factors, Occupancy schedules and residential loads and take into account inflation and NPV in the calculation of savings.

\section{Methodology for the validation of thermal simulations of a real building}

This document shows a methodology to validate thermal simulations of a real building. Comparison metrics and tolerance ranges, both obtained from the literature, are also included. An IER building is used as a case study. Two cases are proposed, a base case where sun protection is used and another one where it is not used. The results include graphs for qualitative comparison of the measurement and the two simulated cases. A quantitative comparison was made using the most common metrics described in the literature review. In both comparisons, the result obtained for case 2 is better than for case 1, this suggests that the solar protections of the building are absorbing heat, due to the red color, and are transmitting heat by conduction and radiation to the interior of the building.


Important elements to consider: It is likely that for the ZEB thesis data will need to be collected from the building of interest, for which a correct validation will need to be carried out. This thesis can be a good guide in case I don't remember very well what the methodology was like.


\section{Thermal comfort studies}

The aim of this document is to explore the evaluation of thermal comfort that the use of passive strategies and low energy consumption provides to occupants in naturally ventilated buildings. In this work, a methodology is proposed for the validation of thermal simulations, whose results show that the building model obtained with the calibration process is applicable to different times of the year, occupancy and ventilation conditions, as well as obtaining accurate results of the thermal comfort assessment using night ventilation strategies and change in the absorptivity of the envelope. The thesis also proposes a methodology for the evaluation of thermal, acoustic and light comfort in the design stage of the new IER building, for which it is proposed to use the extended predicted mean vote method.


Important elements to consider: The validation methodology of the bachelor thesis is used. The document contains important information about the PMVe.



