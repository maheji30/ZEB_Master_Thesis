\chapter{Introduction}
\label{chap:introduccion}

\section{General objective}

To implement a methodology to evaluate zero energy buildings in Mexico.


\section{Specific objectives}

\begin{itemize}

\item To provide a literature review about the concept of NZEB (Net-Zero Energy Buildings).

\item Evaluate the NZEB definitions using a numerical model of the IER's new building.

\item To Implement the PMV and ePMV within the methodology.

\item To propose and evaluate strategies for passive design, energy efficiency and renewable generation that allow meeting the NZEB definitions.

\item To propose a guide that can be taken as a reference in future building construction projects at Mexico  so that they can achieve NZEB status.

\end{itemize}



\section{Energy Buildings: A Critical Look at the Definition}
According with \citep{torcellini2006zero}, a Net Zero-Energy Building (NZEB) is a construction that uses efficiency gains in order to reduce the building's energy consumption so the rest of the energy needs can be supplied by renewable technologies. Due the renewable generation largely depends on the environmental conditions of the place, it is difficult to ensure that the technology will always produce the necessary energy to satisfy the demand of the building; sometimes there will be more generated energy than is needed and sometimes there will be less generated energy than is needed. For that reason, in order to ensure the constant energy supply for buildings, it is necessary that NZEB are connected to the grid, so when the generation is greater than the building's loads, excess electricity is exported to the grid and it can offset later energy use. The solution of this situation could be the energy storage of the generated energy excesses, but storage technologies still being limited and expensive, so the use of the grid is substantial to achieve a NZEB.

\citep{} propose four definitions for NZEB: Net Zero Site Energy (site-NZEB), Net Zero Source Energy (source-NZEB), Net Zero Energy Costs (cost-NZEB) and Net Zero Energy Emissions (emission-NZEB). Before describing the definitions it is important to know about Source energy, which refers to the primary energy used to generate and deliver the energy to the site. To meet site-NZEB definition, the building's renewable generation must be equal or greater than the total energy consumption. A limitation of this definition is that source energy values of the fuels are not considered, it is the same one energy unit of electricity than one energy unit of natural, when is very well known that electricity generation and distribution process involves more energy than in natural gas. As in the previous definition, to meet source-NZEB definition, the building's renewable generation must be equal or greater than the total energy consumption too, but in this case, source energy values are considered, so it is not the same one energy unit of electricity than one energy unit of natural gas. To calculate a building's total source energy, generated and consumed energy is multiplied by the appropriate site-to-source conversion factors (StSF). To meet cost-NZEB, the earned money by exporting renewable electricity to the grid must be equal or greater than the spent money by using electricity and natural gas in the building. Finally, to meet emission-NZEB, the building saving emissions must be equal or greater than produced emissions by using fossil energy sources. The use of each definition depends on the project goals that the building owner has. 



\section{Extension of the PMV model to non-air-conditioned buildings in warm climates}
\citep{} developed the Predicted Mean Vote(PMV) model to anticipate the thermal sensation mean vote of occupants in a space with a seven point scale, where -3 means a very cold sensation, 0 is a comfortable sensation and +3 is a very hot sensation. The PMV model agrees well in buildings with Heating, Venting and Air-conditioning (HVAC) systems, situated in cold, temperate and hot climates, during both summer and winter. The disadvantage of the model is that, when it is used in buildings with no HVAC systems, the result of the prediction is overestimated, in other words, the model predicts a thermal sensation vote warmer than it really is.

This phenomenon occurs due the psychological adaptation of the occupants, which are people who are used to live in hot climates, so they are likely to judge a hot environment as less unacceptable than people who are used to HVAC systems. \citep{} propose that this psychological adaptation can be expressed by an expectation factor, \textit{e}. This factor can take values between 0.5 and 1, where 1 is the appropriate value for buildings with HVAC systems. For buildings without HVAC systems, the expectation factor depends on the duration of hot climate during the year and whether there are other buildings in the region with HVAC systems. Therefore, if the building is in a region with hot climate all the year and there are few or no other buildings with HVAC systems, \textit{e} may be 0.5 and it can be 0.7 if there are many other buildings with air-conditioning. 

A second factor that contributes to this overestimating is the physiological adaptation. In many field studies in office, metabolic rate is estimated with a survey that identifies the percentage of time the person was sitting, standing, or walking. This method doesn't acknowledge the fact that people unconsciously tend to slow down their activity when they feel hot. They adapt to the hot environment by slowing down their metabolic rate. Therefore, a reduced metabolic rate should be inserted in the PMV calculation. \cite{} created a global database of thermal comfort field experiments, where they identified that metabolic rates are reduced by 6.7\% for each PMV scale unit above neutrality. Once the reduced metabolic rate is obtained, it must be used to recalculate the PMV, which is finally multiplied by the appropriate \textit{e}. This method, where psychological and physiological thermal adaptations are acknowledged to improve the accuracy of PMV in building without HVAC systems, is called extended Predicted Mean Vote (ePMV).


\section{Lessons Learned from Case Studies of Six High-Performance Building}
NREL studied six buildings to understand the issues related to the design, construction, operation, and evaluation of the current generation of low energy commercial buildings. Each case study developed a list of lessons learned and recommendations relevant to that unique building. These buildings and the lessons learned from them help inform a set of best practices, beneficial design elements, technologies, and techniques that should be encouraged in future buildings, as well as pitfalls to be avoided. 

The 30\% of total energy in the commercial buildings is for lighting, so there are two practices that were applied in the six buildings in order to achieve energy savings in this end use. The first is by daylighting design, which means that is necessary to design strategically the building so it can allow the access of daylight and reduce the consume of electric light. In the six buildings, this was done through the use of clerestories, elongated east-west axes, south glazing, skylights and sidelit and toplit daylight. The second practice is daylight and lighting controls, which means that it is necessary to install the appropriate equipment in order to turn off electric lighting in response to natural light or when there is no occupancy. The six buildings used daylight sensors and motion sensors. For Oberlin, Zion, TTF, and BigHorn, the total lighting saving represented the largest source of the total energy saving. Total lighting energy savings ranged from 93\% in the BigHorn warehouse to 30\% at CBF. 

Natural ventilation can also provide significant energy savings. Four of the buildings used natural ventilation for cooling and ventilation. The systems at Zion and BigHorn were mostly successful. Zion was designed to be entirely naturally ventilated with a stack effect that is primarily driven by cool air provided by the cooltowers. The EMS mechanically operates windows in the clerestory. Natural downdraft cooltowers that require no fans are also considered natural ventilation. Lower manually operated windows complete the scheme. BigHorn has motor-actuated clerestory windows that are controlled either by the EMS or by wall switches. The front doors, which open during normal business hours in the summer, provide supply air to complete the scheme. 

The use of PV systems to meet the ZEB goals is substantial. During night-time hours when the Oberlin or Cambria PV system was in standby mode, the inverters consumed electricity. This losses accounted for 7\% to 37\% of the total PV output. An automatic circuit that disconnects the PV system from the grid when the PV system is down and reconnects when the PV system is operational should be implemented. Also PV panels should be located where they will not be shaded. Each PV array experienced shading, resulting in a range of energy penalties from 2\% to 44\% output reduction.

\section{Criteria of sustainable construction - UNAM}
In September 2020, the General Direction of Works and Conservation of UNAM published this document as a guide that establishes technical, preventive, corrective and safety measures in the construction and remodelling of university buildings, with the aim of minimizing the negative effects that impact the environment, taking advantage of natural resources and elements in a sustainable manner.  The objective of this document is to provide the UNAM dependencies with guidelines for proper management in the use, exploitation and saving of water and electricity from the project and during the construction and operation of the buildings. The document contains important information about the buildings at CU-UNAM, such as lighting levels, dimensions of sanitary and education spaces, characteristics of the envelope and the standards involved.

\section{Net-Zero Energy Buildings: the influence of definition on greenhouse gas emissions}

This study compares the effectiveness of the four NZEB definitions to reduce operational emissions of a building. A simple geometry located in two different cities, that is, Toronto and Miami, with four different energy behaviours, which are simulated in OpenStudio, is used. It was found that, for both locations, the use of a cost-NZEB balance leads to lower emissions, a reduction of 102-145$\%$ for Toronto and 99-117$\%$ for Miami. The emission-ZEB definition, contrary to its designation, is the least effective in reducing GHG emissions, at 86$\%$ for Toronto and 89-94$\%$ for Miami.

Important elements to consider: The article shows the most important equations to develop each of the definitions.


\section{A review of net zero energy buildings in hot and humid climates: Experience learned from 34 case study buildings}

This document summarizes the design features and technology choices obtained through analysis of 34 NZEB cases in hot and humid climates around the world. The design features and technologies are divide into five groups: architectural design and envelope; heating, ventilation and air-conditioning (HVAC); lighting; plug load equipment; and renewable energy technologies. 

The case studies indicated NZEBs are always equipped with an advanced HVAC system and energy-efficient ventilation strategies. Twenty-four cases use natural ventilation strategies to introduce free cooling to NZEBs and reduce HVAC system energy use. Solar photovoltaics are the most common renewable energy technologies adopted by NZEBs cases, with 30 cases found in this research.

The study found that passive design and technologies such as daylighting and natural ventilation are often adopted for NZEBs in hot and humid climates, together with other energy efficient and renewable energy technologies. 

The Xingye building case demonstrated that natural ventilation could reduce the building's cooling energy from 2.5 kWh/m2 to less than 0.5 kWh/m2 per month — an 80\% cooling energy reduction in the shoulder seasons compared with the summer season. With daylighting, the Xingye building demonstrated that monthly lighting energy intensity can be controlled less than 0.5 $kWh/m2$ XXX escribir bien.

\section{Ten questions about zero energy buildings: A state of the art review.}
This document has the aim of identifying, developing and understanding the main characteristics of zero energy buildings. For this, a review of the state of the art of the topic is carried out, where 97 articles considered to be of greater relevance were selected, in the period from 2006 to 2020. The methodology consisted of an analysis of these texts based on ten questions formulated to address the topic. The questions refer to definitions (P1), sustainability (P2), technologies involved (P3), emissions (P5), energy (P4) (P6) (P7), regulations (P8), climate change (P9) and future projections (P10). The work allows us to conclude that the ZEBs are integrated in a holistic way in the transformation towards a renewable and sustainable future in terms of energy solutions and have the potential to be implemented in different geographical and climatic positions.


Important elements to consider: Consumption for heating and cooling is usually the highest in buildings, so the reviewed literature proposes achieving savings in ZEB, between $\%$25 and $\%$50, respectively, limiting both at 30 $\frac{kWh}{m^{2}*year}$.


\section{A Guide to Zero Energy and Zero Energy Ready K–12 Schools}

This document describes the steps to create a ZE school. These steps serve as a guide to ensure that a school achieves its ZE design goal and maintains its ZE status after it is occupied and operational. The topics covered in the steps are: Assessment of the needs of a building, integration of zero energy goals in the design, management of goals in the design and operation phases, performance evaluation, etc. Each step also includes an experience about the step. These brief anecdotes were provided by the participating school districts and offer a brief synopsis of the district's experience with that step.


\section{Analysis of the environmental, energy and economic effects between a sustainable construction model and a conventional one in Querétaro}

The document focuses on performing an analysis of the environmental, energy and economic effects resulting from the construction of buildings using a sustainability model compared to a conventional model, in order to obtain a cost indicator when using a sustainable construction model to a house and see if it is possible that it is not more than 15$\%$ XXX unifica estilos  against using a traditional construction model. 3 traditional housing models and 3 sustainable housing models were proposed to make comparisons between them. The energy and thermal simulation of all the cases was carried out in EnergyPlus. Comparing the best-equipped sustainable housing model with the worst-equipped traditional housing model, the percentage increase in cost was 14.75\$%$.

Important elements to consider: Site-to-Source conversion factors, CO2 emission factors, Occupancy schedules and residential loads and take into account inflation and NPV in the calculation of savings.

\section{Methodology for the validation of thermal simulations of a real building}

This document shows a methodology to validate thermal simulations of a real building. Comparison metrics and tolerance ranges, both obtained from the literature, are also included. An IER building is used as a case study. Two cases are proposed, a base case where sun protection is used and another one where it is not used. The results include graphs for qualitative comparison of the measurement and the two simulated cases. A quantitative comparison was made using the most common metrics described in the literature review. In both comparisons, the result obtained for case 2 is better than for case 1, this suggests that the solar protections of the building are absorbing heat, due to the red color, and are transmitting heat by conduction and radiation to the interior of the building.


Important elements to consider: It is likely that for the ZEB thesis data will need to be collected from the building of interest, for which a correct validation will need to be carried out. This thesis can be a good guide in case I don't remember very well what the methodology was like.


\section{Thermal comfort studies}

The aim of this document is to explore the evaluation of thermal comfort that the use of passive strategies and low energy consumption provides to occupants in naturally ventilated buildings. In this work, a methodology is proposed for the validation of thermal simulations, whose results show that the building model obtained with the calibration process is applicable to different times of the year, occupancy and ventilation conditions, as well as obtaining accurate results of the thermal comfort assessment using night ventilation strategies and change in the absorptivity of the envelope. The thesis also proposes a methodology for the evaluation of thermal, acoustic and light comfort in the design stage of the new IER building, for which it is proposed to use the extended predicted mean vote method.


Important elements to consider: The validation methodology of the bachelor thesis is used. The document contains important information about the PMVe.



